\newglossaryentry{api}
{
    name={API},
    description={Application Programming Interface}
}
\newglossaryentry{baas}
{
    name={BaaS},
    description={Backend as a Service: Plattform, die die Bereitstellung von Backend-Funktionalität ermöglicht inklusive Datenbank, Authentizifierung, Datei-Upload, etc}
}
\newglossaryentry{ui}
{
    name={UI},
    description={Benutzeroberfläche, der Teil der Anwendung, der für den Nutzer sichtbar und nutzbar ist. Englisch: „User Interface“}
}
\newglossaryentry{bug}
{
    name={Bug},
    description={Fehler in der Software}
}
\newglossaryentry{bugfix}
{
    name={Bugfix},
    description={Behebung eines Fehlers in der Software}
}
\newglossaryentry{mui}
{
    name={MUI},
    description={Material-UI (Material Design Komponenten für React)}
}
\newglossaryentry{ci}
{
    name={CI},
    description={Continuous Integration: Frühes Integrieren kleiner Änderungen in den Hauptzweig (Git Branches)}
}
\newglossaryentry{cd}
{
    name={CD},
    description={Continuous Delivery / Continuous Deployment: Automatisches Ausrollen der neuen Funktionalität}
}
\newglossaryentry{jsx}
{
    name={jsx},
    description={Javascript XML: Erweiterung von Javascript, die es ermöglicht, HTML-Elemente in Javascript zu definieren}
}
\newglossaryentry{tsx}
{
    name={tsx},
    description={TypeScript XML: Erweiterung von TypeScript, die es ermöglicht, HTML-Elemente in TypeScript zu definieren}
}
\newglossaryentry{react}
{
    name={React},
    description={Frontend-Framework basierend \gls{node}, das von Facebook entwickelt wird. Es ermöglicht die Entwicklung von Benutzeroberflächen für Webanwendungen}
}
\newglossaryentry{node}
{
    name={Node.js},
    description={Javascript-Plattform, die es ermöglicht, Javascript außerhalb des Browsers auszuführen}
}
\newglossaryentry{confluence}
{
    name={Confluence},
    description={Software zur Zusammenarbeit und Dokumentation}
}
\newglossaryentry{jira}
{
    name={Jira},
    description={Software zum Projektmanagement}
}
\newglossaryentry{usp}
{
    name={USP},
    description={Unique Selling Point}
}
\newglossaryentry{usability}
{
    name={Usability},
    description={Bedienbarkeit oder Nutzbarkeit einer Anwendung}
}
\newglossaryentry{wireframe}
{
    name={Wireframe},
    description={Grobkonzept einer Benutzeroberfläche}
}


\glsaddall
