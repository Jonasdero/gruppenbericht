\chapter{Projekt}
\label{ch:project}

Zu aller erst wird das Projekt vorgestellt und die Aufgabenstellung erläutert. Dabei wird auch auf die Ausgangssituation eingegangen und das Projektziel definiert.
Dadurch wird der Leser in die Lage versetzt, die weitere Projektdokumentation besser zu verstehen.
Anschließend wird das Team vorgestellt und die einzelnen Rollen und Aufgaben der Teammitglieder erläutert.

Die nachfolgende Projektbeschreibung stammt direkt aus dem Projektaushang und wurde nur leicht angepasst. Die Projektbeschreibung ist in der Projektarbeit nicht weiter ausführlich dargestellt, da sie nur eine kurze Einführung in das Projekt darstellt.

\section{Projektbeschreibung}
\label{sec:project-description}

Das Projekt „Digital Home Town“ ist ein „smart city-Konzept“, welches zur digitalen
Vernetzung verschiedener Generationen und Interessenten im Sozialraum dient. Lebendiger
Austausch, voneinander lernen, kommunale Ressourcen effizient nutzen, als Stadt virtuell
zusammenwachsen, darum geht es in diesem Projekt.

\subsection{Ausgangssituation}
\label{sub:project-start}

Grundsätzlich geht es darum eine Plattform zu bauen, die Zielgruppen- und generationenübergreifend eine Stadt vernetzt und verschiedene Features anbietet.
So können virtuelle Kursräume Informationen zur Verfügung stellen, durch Chattools Diskussionsforen eingerichtet werden und die physikalische Infrastruktur treffend verteilt werden, z. B. durch ein Buchungstool (Turnhallenbelegung, Sportheim).
So können z. B. schulische Inhalte auch anderen Generationen zur Verfügung gestellt werden und umgekehrt zeithistorische Informationen für Schüler dargeboten werden sowie kommunale Ressourcen optimal genutzt und ausgelastet werden.
Ferner entsteht ein digitaler Marktplatz für lokale Betriebe, um die Wirtschaftskraft in der Region zu stärken.
Diese Beispiele stellen nur exemplarische Ansätze und Features dar.
Denn die Plattform wächst in einer organischen Evolution mit den Anforderungen der nutzenden Gesellschaft.

Zwei besondere Anforderungen bringt dieses Projekt mit sich:
\begin{itemize}
  \item Das fertige Produkt muss von Beginn an generationenübergreifend eine hohe Userakzeptanz
  erlangen, indem die Zugänglichkeit und Bedienbarkeit leicht, schnell und barrierefrei
  gewährleistet ist.
  \item Um ein solches Projekt langfristig zu realisieren, müssen Schnittstellen modular mitgedacht werden, damit auch sich verändernde Bedarfsstrukturen berücksichtigt werden können.
\end{itemize}

\subsection{Projektziel}
\label{sub:project-goal}

Ziel ist es in diesem Projekt eine prototypische Plattform zu entwickeln, die in ihrer
Multifunktionalität verschiedenste Anforderungen für die Nutzer erfüllt. Ob es nun …

\begin{itemize}
  \item[\dots] ein Planungstool, ähnlich einem mit anderen Usern geteilten Kalender,
  \item[\dots] ein Chat oder Blog-Tool zur virtuellen Interaktion und diskursiven Auseinandersetzung ist,
  \item[\dots] virtuelle Kursräume und Datenarchive sind, die generationenübergreifendes Lernen ermöglichen,
  \item[\dots] eine Tauschbörse oder einfach nur
  \item[\dots] eine originäre Homepage ist,

\end{itemize}
mit allen Features wird das Zusammenleben in einer Kommune leichter und dem
aktuellen Stand der Technik gerecht. Deshalb sollen diese sämtlich in diese eine Plattform
einbezogen werden.

\subsection{Aufgaben}
\label{sub:project-tasks}
Die zu bearbeitenden Aufgaben umfassen folgende Punkte:
\begin{itemize}
  \item Einbindung des Kunden und Abstimmung zu spezifischen Anforderungen
  \item Anforderungsanalyse der zu priorisierenden Features
  \item Konzeption einer multimedialen Plattform
  \item Entwicklung eines Rechte- und Rollenkonzepts
  \item Entwicklung einzelner Features
  \item Umsetzung der Spezifikationen im Sinne der Entwicklung eines lauffähigen Prototyps.
  \item Konsequente Berücksichtigung der Usability für alle Adressaten
  \item Test der lauffähigen Funktionalitäten.
\end{itemize}

\section{Team}
\label{sec:project-team}

Im Rahmen des Projekts „Digital Home Town“ arbeiteten folgende Personen zusammen:

\begin{itemize}
  \item \textbf{Daniel Reitberger} (Scrum Master)
  \item \textbf{André Reif} (Product Owner, Entwickler)
  \item \textbf{Rebecca Vogler} (Product Owner, Tester)
  \item \textbf{Boas Dünkel} (Entwickler)
  \item \textbf{Jan Scholz} (Entwickler)
  \item \textbf{Jonas Roser} (Entwicker)
\end{itemize}

Dieses Team hat sich selbstständig für das Thema „Smart City“ entschieden und wird von Herrn Johannes Walser betreut.
Dieser ist der Ansprechpartner für alle Fragen und Anliegen, die sich im Projekt ergeben.
Dazu war er der „Kunde“ des Projekts, also derjenige, der die Anforderungen an das Produkt definiert hat.

Alle Teammitglieder haben sich in der Projektarbeit gegenseitig unterstützt und sich gegenseitig weiterentwickelt.
