% \section{Sprintoverview}

\subsection{Sprintübersicht}

\subsubsection{Sprint 0}
Der „nullte“ Sprint stellt den Start des Projektes dar. Nachdem die Rollen innerhalb des Scrum Teams festgelegt und deren Aufgaben definiert wurden, wurden Kommunikations-, Management- und Dokumentationsregeln festgelegt. Es wurde eine Wettbewerbsrecherche und eine Ideensammlung ausgearbeitet, sowie die Vision für das Projekt überlegt.

\subsubsection{Sprint 1}
Im ersten Sprint wurde die Vision weiter ausformuliert und Ideen zur Umset-zung der Vision gesammelt. Parallel dazu hat sich das Entwicklerteam auf die Verwendung von Technologien geeinigt bzw. eingearbeitet werden. Daraus entstand ein erstes Architekturkonzept für die Plattform. Anhand der gesammelten Ideen konnten bereits erste User-Storys erstellt werden. Die Kundenanforderungen sowie die Zielgruppe wurden mit Hilfe des Value Proposition Canvas identifiziert und analysiert.

\subsubsection{Sprint 2}
Das Ergebnis des zweiten Sprints umfasste bereits ein bestehendes Grundgerüst der Website. Die Registrierung sowie eine Seite in der eine Feature Übersicht zu sehen war wurden bereits umgesetzt. Über Personas wurde das Kundensegment weiter spezifiziert und daraus Features abgeleitet. Zeitgleich wurden als elementare Elemente der Plattform Private und Vereinsprofile spezifiziert.

\subsubsection{Sprint 3}
Während des Sprint 3 wurde die Struktur und Oberflächenarchitektur der Website geplant, wobei der Schwerpunkt auf dem Basisfeature „Profil“ lag. Die zentrale Frage lautete, wie ein Benutzerprofil in unserem Netzwerk aussehen sollte und welche Inhalte es enthalten sollte. Das Ergebnis war eine Liste aller Anforderungen, die für die Erstellung eines Benutzerprofils erforderlich sind. Es wurde eine Übersicht über alle Features erstellt. Die Funktionen, die sich im vorherigen Sprint überlegt wurden, wurden kategorisiert und priorisiert. Auch die Profile von Vereinen und deren Besonderheiten wurden in Betracht gezogen.

\subsubsection{Sprint 4}
Im vierten Sprint wurden die zuvor besprochene Oberflächenarchitektur und Seitenstruktur in Wireframes umgesetzt. Dies ermöglichte es, erste Designentscheidungen zu treffen und die Inhalte und Funktionen unserer Website ge-nauer zu diskutieren. Durch die in Sprint 0 angesprochene Wettbewerbsanalyse konnten wir das Alleinstellungsmerkmal für unsere Plattform „Digital Dahoam“ identifizieren: Das Tagsystem. Nun konnten Anforderungen für das Dashboard und eine Filterfunktion über Tags festgelegt werden. Gleichzeitig wurde das Anlegen von Benutzerprofilen entwicklungsseitig umgesetzt.

\subsubsection{Sprint 5}
Als weitere Basisfunktion vom Digital Home Town wurde die Chatfunktion rudimentär umgesetzt. Im Sprint wurde ebenfalls and er Weiterentwicklung des Layouts gearbeitet und erste Aspekte der Benutzerfreundlichkeit in die Entwicklung einbezogen. Durch den Ansatz des User-Story-Mapping wurde eine Übersicht als Grundlage für die Projektplanung erarbeitet. Zusätzlich wurden weitere Funktionen für Vereine definiert und priorisiert.

\subsubsection{Sprint 6}
Im Sprint 6 wurde einerseits die Beitragsfunktion implementiert. Andererseits wurden Texte und Bilder für die Landingpage definiert, um sowohl interessierte Privatpersonen als auch Vereine über die Inhalte und Möglichkeiten innerhalb von Digital Dahoam zu informieren. Darüber hinaus wurde die Benutzerfreundlichkeit mit bestehenden Plattformen abgeglichen. Wireframes für das Profil und das Dashboard wurden weiter ausgearbeitet. Ein Konzept für einen Feldtest unserer bestehenden Plattform wurde ebenfalls erarbeitet.

\subsubsection{Sprint 7}
Während des siebten Sprints wurden grundlegende Accounteinstellungen implementiert und der Registrierungsprozess weiter verbessert. Die im vorherigen Sprint entwickelten Beiträge wurden interaktiver gestaltet. Um Feedback zum bestehenden System zu erhalten und Features besser priorisieren zu können, wurde eine Usability-Umfrage erstellt und ausgewertet. Darüber hinaus wurden Layouts für zuvor nur funktional umgesetzte Funktionen definiert.

\subsubsection{Sprint 8}
In diesem Sprint wurden das Merkzettel-Feature sowie der Marktplatz implementiert. Bestehende Funktionen wurden zudem weiter optimiert und getestet. Zur Befüllung des Netzwerks wurden Dummy-Datensätze erstellt, die später im Netzwerk eingespielt werden können. In einem Abstimmungstermin mit dem Kunden wurde der aktuelle Stand von Digital Dahoam vorgestellt und weitere Umsetzungen abgeleitet.

\subsubsection{Sprint 9}
Im neunten und letzten Sprint wurden noch offene Bugs behoben und bereits begonnene Features final umgesetzt. Die Ergebnisse wurden dann im Zuge einer Abschlusspräsentation dem Kunden präsentiert. 
