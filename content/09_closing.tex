\chapter{Fazit}
\label{ch:conclusion}

Abschließend lässt sich zum erreichten Ergebnis festhalten, dass wir eine Plattform geschaffen haben, die dem gesetzten Projektziel entspricht. Die Plattform bietet mit ihren einzelnen Features die Möglichkeit der Vernetzung von Nachbarn. Hierbei sind vor allem die generalistisch zu verwendenden Beiträge zu nennen. Mit den Beiträgen erreichen Menschen oder Vereine die nötige Reichweite, um wichtige Informationen preiszugeben. Des weiteren ist es durch die Möglichkeit etwas anzubieten (Geben) bzw. Anfragen zu stellen (Nehmen) möglich, das Menschen in der Nachbarschaft in Kontakt kommen bzw. sich über die Plattform kennenlernen können.

Die Benutzbarkeit der Plattform ist durch die intuitive und übersichtliche Benutzeroberfläche gegeben. Die Benutzeroberfläche ist hierbei in erster Linie für Desktop-Geräte optimiert, kann jedoch aufgrund der ausgewählten Technologie mit begrenztem Aufwand für mobile Geräte angepasst werden.

Durch die Verwendung von Firebase, das von Google betrieben wird, sowie von Vercel zum Hosting der Plattform, ist eine sehr hohe Verfügbarkeit der Plattform gegeben.

Auch wenn es Mangels der Zeit nicht möglich war, alle Features in einer optimalen Qualität zu implementieren, haben wir es in der knappen Zeit geschafft, eine funktionierende Plattform zu entwickeln, die in iterativen Schritten immer weitere Features dazugewonnen hat. Die Aufgabenstellung an sich hätte sicherlich eine Vielzahl weiterer Features ergeben, umso wichtiger war unsere Diskussion, welche Features für die Plattform im ersten Schritt wirklich notwendig sind. Ich finde, unsere Plattform entspricht diesem Prinzip und kann sich deshalb sehen lassen.
