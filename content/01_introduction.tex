\chapter{Einleitung}
\label{ch:introduction}

Dieser Projektbericht beschreibt die Entwicklung einer prototypischen Plattform für einen interaktiven Austausch innerhalb eines Ortes, einer Stadt oder einer Gemeinde.
Er wurde im Rahmen des Masterstudiengangs \textit{Software-Engineering} an der Technischen Hochschule Georg Simon Ohm Nürnberg erstellt.

Zu Beginn wird auf die Beschreibung des Projekts eingegangen, die Anforderungen an das Produkt werden definiert und die Aufgaben des Projekts werden aufgezeigt.
Damit wird eine Grundlage gelegt, die für die weitere Arbeit im Projekt benötigt wird. Außerdem wird das Team in seinen verschiedenen Rollen vorgestellt.

Im nächsten Kapitel wird der Prozess der Entwicklung der App beschrieben.
Dies geschah über eineinhalb Jahre hinweg und umfasst die Planung, die Konzeption und die Implementierung der Plattform.
Mit der agilen Softwareentwicklungsmethode Scrum wurde ein iterativer Prozess durchgeführt, der die Entwicklung der Anwendung in mehreren Sprints abwickelte.

Im vierten Kapitel werden technische Aspekte von „Digital Dahoam“ beschrieben.
Hierbei werden Technologien und Frameworks vorgestellt, die für die Entwicklung der Plattform verwendet wurden.
Diese werde jeweils mit einem kurzen Überblick vorgestellt, der genaue Zweck bzw. Einsatz im Projekt erklärt und die Entscheidung für die Verwendung dieser Technologien wird erläutert.

Die Architektur der Plattform wird im fünften Kapitel genauer beschrieben und die einzelnen Komponenten vorgestellt.
Außerdem wird die Kommunikation zwischen den Komponenten beschrieben.

Details zur Umsetzung der Website erscheinen in Kapitel \ref{ch:implementation}.
Hierbei wird die Umsetzung der einzelnen Funktionen der Plattform beschrieben.
Zunächst wird die Umsetzung der Benutzeroberfläche beschrieben, danach werden die einzelnen Funktionen der Plattform vorgestellt.
Außerdem wird durch Screenshots die Benutzeroberfläche der Plattform dargestellt.

Anschließend werden die Aktivitäten in den Sprints beschrieben.
Dabei wird kurz über die einzelnen User Stories berichtet, die in den Sprints umgesetzt wurden.
Außerdem wird die Arbeit im Team beschrieben und die Ergebnisse der einzelnen Sprints zusammengefasst.

Der Bericht wird abgeschlossen mit einem Fazit, in dem die Ergebnisse des Projekts zusammengefasst werden.
Außerdem wird ein Ausblick auf mögliche Weiterentwicklungen der Plattform gegeben und eine Bewertung der Arbeit im Projekt abgegeben.
